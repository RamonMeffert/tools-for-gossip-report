% !TEX root = 0_report.tex

\section{Discussion}

\subsection{Survey}

While the survey yielded useful results, its exploratory nature along with the low number of respondents mean that its results are difficult to quantitatively analyse.
Therefore, it could be interesting to perform a larger user study.
This could include a larger survey focused more on objective analysis of the tool's ergonomics.
Another option would be to perform a number of cooperative walk-throughs to gain more insight in problems users might face.

\subsection{Further research}

At the time of publication of this report, several planned features -- some part of the initial concept, some arising from the results of the survey -- have not been implemented yet. For completeness' sake, we will list these features here shortly:

\begin{itemize}
    \item The custom protocol creation (Section~\ref{sec:custom-protocols}) does not work yet;
    \item The rendered graph sometimes jumps when changing it;
    \item User input is not persisted; when the page is refreshed, all input is gone.
\end{itemize}

Besides these features, a more up-to-date list of smaller planned features and bug fixes can be found at \url{https://github.com/RamonMeffert/tools-for-gossip/issues}.

\subsubsection{Possible extensions of the tool}

The current version of the tool allows user to model dynamic gossip protocols as defined in \textcite{van_ditmarsch_dynamic_2018}.
However, there is more to gossip than what is described in that paper.
In this section, other papers on (dynamic) gossip are presented which introduce additional concepts.
To make the tool more broadly usable, it could be interesting to add some or all of these aspects to the tool.

The first possibility considered here is that of meta-knowledge or epistemic information, such as in \textcite{herzig_how_2017}.
That means that agents can not only know secrets and numbers, 
but also whether other agents know a secret or number.
This allows for more complex protocols.
An example of this can be found in the \texttt{GoMoChe} tool \parencite{gattinger_m4lvingomoche_2020}.

Another idea is to include a temporal aspect, such as in \textcite{slavkovik_temporal_2019}
(note that this paper also includes epistemic information).
This means that agents cannot always be reached.
The paper gives as examples for this problem communication between geostationary satellites that might not always be able to `see' each other,
or a phone with a battery that sometimes does not have enough charge to communicate.
This might be interesting, because this means that instead of calls being either always possible or always impossible, 
they might \textit{sometimes} be possible or impossible as well.

A related problem is introduced by \textcite{martins_dealing_2020}: 
normally, it is assumed that agents in dynamic gossip are reliable and tell the truth about the secrets they hold.
However, if agents lie, many known properties of dynamic gossip suddenly no longer hold.
By introducing unreliable agents, a new goal is added to the gossip system:
\textit{identify the unreliable agents}.
An implementation of this (though slightly different from the version described in the paper mentioned before) can be found at \textcite{van_den_berg_unreliable_2020}.

Another modification to secrets could be taken from \textcite{demers_epidemic_1988}, who propose a system in which all secrets have a `hotness'.
This `hotness' indicates how often the secret has been spread.
This can then be used as a heuristic to determine whether it is useful to communicate the secret.
This might be interesting because in the current system, an agents communicates all their secrets in a call.
However, in a real-life application, this might not be desired:
In large networks, communicating all secrets all the time might lead to large amounts of data being sent even though this might be data the agents receiving the secrets already have.

Besides these improvements to the functionality of the tool, 
another possible improvement would be to make it suitable for usage on mobile devices such as smartphones and tablets.
According to \textcite{cisco_cisco_2020}, the number of mobile devices used to access the internet is larger than the number of desktop devices, and is still increasing.
This increase is also most visible among younger people,
and since a key user group of this tool is intended to be students,
having the tool be usable on a mobile device will probably increase the likelihood of it being used.