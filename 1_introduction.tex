% !TEX root = report.tex

\section{Introduction}\label{sec:introduction}

\textit{Gossip protocols} are protocols that describe the way rumors---or, more generally, secrets---are shared in multi-agent environments.
The goal of the protocols is to communicate all secrets to all agents.
A lot of research has been done in this field, 
starting with research on the spread of infectious diseases \parencite{kermack_contribution_1927}.

The definition of the gossip problem generally used nowadays was first introduced in \citeyear{hajnal_cure_1972} by \citeauthor{hajnal_cure_1972}.\footnote{TODO: Tijdeman (1971) might have been earlier, but I have not been able to find a pdf of that paper. It is referenced in \Textcite{van_ditmarsch_dynamic_2018} though.}
In short, agents are represented as nodes in a graph, with the edges representing a ``call''---%
that is, one agent transferring all of their secrets to another agent.
When all agents can contact all other agents, \citeauthor{hajnal_cure_1972} proved that this can be done in \(2n-4\) calls, where \(n\) is the number of agents.

The problem as formulated above requires the oversight of a central authority in order to know whether all agents know all secrets.
However, there are many applications where this is not feasible or desirable\footnote{TODO: maybe find a citation for this? Or just explain it better}.
Another problem is that it often cannot be guaranteed that all agents can contact all other agents.
This has led to the sub-fields of \textit{distributed gossip}, addressing the first issue, and \textit{dynamic gossip}, addressing the second.
The combination of these fields, where there is no overseer and not all agents can contact all other agents, is called \textit{distributed dynamic gossip}.



% - gossip protocols
% - reference some papers indicating origin of problem
% - explain dynamic gossip
% - examples of applications
% - rationale for building a tool
%   - visualise protocol execution
%   - exploration of new protocols


\subsection{Notation}

This notation used in this paper is based off of the notation used in \Textcite{van_ditmarsch_dynamic_2018}
