% !TEX root = report.tex

\section{Introduction}
\label{sec:introduction}

\textit{Gossip protocols} are protocols that describe the way rumors---or, more generally, secrets---are shared in multi-agent environments.
The goal of the protocols is to communicate all secrets to all agents.
A lot of research has been done in this field, 
starting with research on the spread of infectious diseases \parencite{kermack_contribution_1927}.

The definition of the gossip problem generally used nowadays was first introduced in \citeyear{hajnal_cure_1972} by \citeauthor{hajnal_cure_1972}.\footnote{TODO: Tijdeman (1971) might have been earlier, but I have not been able to find a pdf of that paper. It is referenced in \Textcite{van_ditmarsch_dynamic_2018} though. According to Faustine Maffre at \href{https://github.com/FaustineMaffre/GossipProblem-PDDL-generator}{her github} the problem was formulated in 1970 in an unpublished work.}
In short, agents are represented as nodes in a graph, with the edges representing a relation.
This relation can be either the number relation \(N\), representing that one agent has the ``phone number'' of another agent,
and the secret relation \(S\), representing that one agent knows the secret of another agent.
\Citeauthor{hajnal_cure_1972} proved that this can be done in \(2n-4\) calls, where \(n\) is the number of agents, if all agents know the numbers of all other agents---that is, the number relation \(N\) is \textit{complete}.

The problem as formulated above requires the oversight of a central authority in order to know whether all agents know all secrets.
However, there are many applications where this is not feasible or desirable \addcite \footnote{TODO: maybe find a citation for this? Or just explain it better}.
Another problem is that it often cannot be guaranteed that all agents can contact all other agents.
This has led to the sub-fields of \textit{distributed gossip}, addressing the first issue, and \textit{dynamic gossip}, addressing the second.
The combination of these fields, where there is no overseer and not all agents can contact all other agents, is called \textit{distributed dynamic gossip}.

% TODO: explain protocols here

This paper will describe a tool that has been developed to make it easier to explore distributed dynamic gossip.
% ? why? van ditmarsch (2018) states "We do not know approaches that combine dynamic gossip and distributed gossip for 
% ? protocols other than making random calls (i.e., protocol ANY)." so new protocol development might be a reason
It is able to visualise the connections between agents, allows exploring the execution tree of different (semi-arbitrary) protocols and validate call sequences given a graph and/or a protocol.

\subsection{Earlier work}

\subsubsection{Applications of gossip protocols}

Gossip protocols have been in use in applications ranging from epidemics simulations \addcite to managing membership and detecting failure in distributed (database) processes \parencite{decandia_dynamo_2007, das_swim_2002}.
These processes bear some resemblance to distributed dynamic gossip, since membership is managed using gossip protocols:
Whenever a member joins or leaves the network, this is spread using gossip, like the way numbers are shared in dynamic gossip.
The difference lies in the fact that in this application, the secrets and numbers are not shared together per se;
they are exchanged independently.

\subsubsection{Visualisation tools}

A researcher or student interested in dynamic gossip will probably start looking for some way to gain an intuitive understanding of the field.
At the time of writing, a couple of options are available.
However, most of these require either an established understanding of the field and/or proficiency using command-line tools.
In this section, these tools will be discussed in order to highlight some of the aspects the tool proposed in this paper attempts to improve.

The first two tools are tools by Malvin Gattinger (the supervisor of this Bachelor's project):
one works in the browser \parencite{gattinger_webgossip_nodate}, the other is a command-line tool \parencite{gattinger_m4lvingomoche_2020}. 
The first tool, \texttt{WebGossip}, is able to visualise gossip graphs based on user input.
While the input format for that tool is different from the one used in this tool
(for an explanation of the format for the tool described in this paper, see Section~\ref{sec:gossip-graph-visualisation}),
it is similar in that it also takes text input and displays the gossip graph graphically.
The tool also generates code for use in \LaTeX\ papers.
The main differences between this tool and the tool described in this paper can be seen in Table~\ref{tab:tool-comparison}

\begin{table*}
    \centering
    \caption{Comparison between the tool described in this paper and other tools available}
    \label{tab:tool-comparison}
    \begin{tabular}{@{}p{0.3\linewidth}*{3}{M{0.2\linewidth}}@{}}
        \toprule
        Feature 
            & \texttt{GoMoChe} 
            & \texttt{WebGossip} 
            & This paper\\
        \midrule
        Gossip graph visualisation
            & \tpos{Yes}
            & \tpos{Yes}
            & \tpos{Yes}
            \\
        \LaTeX\ output
            & \tneg{No}
            & \tpos{Yes}
            & \tmay{Planned}
            \\
        Call execution
            & \tneg{No}
            & \tneg{No}
            & \tpos{Yes}
            \\
        Call sequence execution
            & \tneg{No}
            & \tneg{No}
            & \tpos{Yes}
            \\
        List allowed calls
            & \tpos{No}
            & \tneg{No}
            & \tpos{Yes}
            \\
        List allowed sequences
            & \tpos{Yes}
            & \tneg{No}
            & \tneg{No}
            \\
        Protocols
            & LNS, CO, CMOWLOG, pig, ANY, NO
            & N/A
            & LNS, CO, wCO, SPI, ANY, TOK, \textcolor{red}{custom}
            \\
        \bottomrule
    \end{tabular}
\end{table*}

Finding call sequences and graph visualisation, command line \parencite{gattinger_m4lvingomoche_2020}

Gossip spreading visualisation, web \parencite{moelker_rrmoelkergossip-visualization_2016}

Gossip graph visualisation, web \parencite{gattinger_webgossip_nodate}

(relevant?) Generating Planning Domain Definition Language code for gossip protocols, command line \parencite{maffre_faustinemaffregossipproblem-pddl-generator_2020}

\subsection{Notation}
\label{sec:notation}

This paper takes its notation from \Textcite{van_ditmarsch_dynamic_2018} and \Textcite{van_ditmarsch_strengthening_2019}.
The notation used is explained below.

\begin{definition}[Gossip graph]
    Let \(A\) be a set of agents \(\{a, b, \dots\}\).
    Two directed binary relations on \(A\) are defined: \(N, S \subseteq A^2\).
    The first denotes the \textit{number} relation, the second the \textit{secret} relation.
    A gossip graph \(G\) is then defined as the a triple \((A, N, S)\).
\end{definition}

\begin{definition}[Relations on gossip graphs]
    
\begin{subdefinition}[Binary relation]
    \(Pxy\) says that agent \(x\) has relation \(P\) to agent \(y\).
    
    Formally: \((x, y) \in P\)
    \label{def:rel-bin}
\end{subdefinition}

\begin{subdefinition}[Identity relation]
    \(I_A\) is the set of relations where all agents have a relation to themselves.
    
    Formally: \(\{(x,x) \mid x \in A\}\)
    \label{def:rel-id}
\end{subdefinition}

\begin{subdefinition}[Converse relation]
    \(P^{-1}\) is the set of relations in \(P\), but with their direction switched.

    Formally: \(P^{-1} = \{(x,y) \mid Pyx\}\)
    \label{def:rel-conv}
\end{subdefinition}

\begin{subdefinition}[Composition relation]
    The composition of the relations \(P\) and \(Q\) is a new relation such that the tuple \((x,z)\) is in said new relation if and only if there exists another agent y such that \((x,y) \in P\) and \((y,z) \in Q\).

    Formally: \(P \circ Q = \{(x,z) \mid \exists y ((x, y) \in P \land (y, z) \in Q) \}\)
    \label{def:rel-comp}
\end{subdefinition}

\begin{subdefinition}
    \(P_x\) represents the agents \(x\) has relation \(P\) with.

    Formally: \(P_x = \{ y \in A \mid Pxy \}\)
    \label{def:rel-acq} % acq for acquaintances ('acquaintance relation' might be a nice name for this relation)
\end{subdefinition}

\begin{subdefinition}
    \(P^i\) represents the \(i\)th composition of relation \(P\) with itself.

    Formally: 
    \[
        P^i = 
        \begin{cases}
            P               & \text{for} \; i = 1\\
            P^{i-1} \circ P & \text{for} \; i > 1
        \end{cases}
    \]
    \label{def:rel-icomp}
\end{subdefinition}

\begin{subdefinition}
    % ? is this description right? It might also be infinite composition
    \(P^*\) represents the set of binary relations that are obtained through repeated relation composition of \(P\) with itself.

    Formally:
    \[
        \bigcup_{i \geq 1} P^i
    \]
    \label{def:rel-star} % not sure what to call this relation
\end{subdefinition}
\end{definition}

\begin{definition}[Calls]
    \label{def:calls}
    A call from an agent \(x\) to an agent y is denoted \(xy\).

    \begin{subdefinition}[Call sequence]
        \label{def:call-sequence}
        Call sequences are denoted using lowercase greek letters, starting with \(\sigma\). \(\epsilon\) denotes the empty sequence. \(\sigma \conc \tau\) denotes the concatenation of two sequences.
    \end{subdefinition}
    
    \begin{subdefinition}[Subsequence]
        \(\sigma_x\) denotes all calls in the sequence \(\sigma\) containing \(x\).

        Formally:
        \[
            \sigma_x = 
            \begin{cases} 
                \epsilon            & \text{if}\; \sigma = \epsilon\\
                \tau_x\mathbin{;}uv & \text{if}\; \sigma = \tau\mathbin{;}uv \land (x = u \lor x = v) \\ 
                \tau_x              & \text{if}\; \sigma = \tau\mathbin{;}uv \land \lnot(x = u \lor x = v) 
            \end{cases}
        \]
    \end{subdefinition}

    \begin{subdefinition}[Effect of a call on a relation]
        \label{def:call-effect-relation}
        \(P^{xy}\) denotes the state of relation \(P\) after call \(xy\).

        Formally: \( P^{xy} = P \cup (\{(x,y),(y,x)\} \circ P) \)
    \end{subdefinition}

    \begin{subdefinition}[Effect of a call on a gossip graph]
        \label{def:call-effect-graph}
        \(G^{xy}\) denotes the state of gossip graph \(G\) after call \(xy\).

        Formally: \( G^{xy} = (A,N^{xy},S^{xy}) \)
    \end{subdefinition}

    \begin{subdefinition}[Effect of a call sequence on a gossip graph]
        \(G^\sigma\) denotes the state of gossip graph \(G\) after call sequence \(\sigma\).

        Formally: 
        \[ 
            G^\sigma = 
            \begin{cases}
                G               & \text{if}\; \sigma = \epsilon\\
                (G^{xy})^\tau   & \text{if}\; \sigma = xy \conc \tau
            \end{cases}
        \]
    \end{subdefinition}

    \begin{subdefinition}[Call sequence possibility]
        \label{def:call-sequence-possibility}
        A call sequence \(\sigma\) is possible on a gossip graph \(G\) iff for all calls \(c_i \in \sigma\) such that \(\sigma = c_i\mathbin{;}\tau\), the call is possible, i.e. \(Nc_i\)
    \end{subdefinition}

    \begin{subdefinition}[Call permissibility]
        \label{def:call-permissibility}
        A call \(xy\) is permitted on a gossip graph \(G^\tau\) with a (possibly empty) history \(\tau\) iff \(\tau\) is possible on \(G\), \(x \neq y\), \(N^\tau xy\) and \(G^\tau \models \pi(x,y)\).
    \end{subdefinition}

    \begin{subdefinition}[Call sequence permissibility]
        \label{def:call-sequence-permissibility}
        A call sequence \(\sigma \conc xy\) is permitted on a gossip graph \(G^\tau\) with a (possibly empty) history \(\tau\) iff \(xy\) is permitted on \((G^\tau)^\sigma\) and \(\sigma\) is permitted on \(G^\tau\).
    \end{subdefinition}
\end{definition}

\begin{definition}[Gossip Protocol]
    \label{def:gossip-protocol}
    A gossip protocol \(\mathsf{P}\) is a combination of a protocol condition \(\pi(x,y)\) and a non-deterministic algorithm \parencite[for examples see][p. 708]{van_ditmarsch_dynamic_2018}.

    \begin{subdefinition}[Protocol condition]
        \label{def:protocol-condition}
        \(\pi(x,y)\) represents a protocol condition, that is, a condition that needs to be satisfied before a call can be made.
        This report takes this condition, like \textcite{van_ditmarsch_dynamic_2018}, as a boolean combination of the following constituents: \(S^\sigma xy\), \(xy \in \sigma_x\), \(yx \in \sigma_x\), \(\sigma_x = \epsilon\), \(\sigma_x = t \conc xz\) and \(\sigma_x = t \conc zx\).
    \end{subdefinition}
\end{definition}

% TODO: finish writing the notation. I have a notes file containing all notation, just have to convert it to latex

% - gossip protocols
% - reference some papers indicating origin of problem
% - explain dynamic gossip
% - examples of applications
% - rationale for building a tool
%   - visualise protocol execution
%   - exploration of new protocols
