% !TEX root = report.tex

\begin{abstract}
The field of gossip theory -- concerned with the way information spreads in networks of agents -- has been the subject of research since at least 1927.
However, with the introduction of new paradigms within this field, such as dynamic and distributed gossip, the field has become more complex.
Information on dynamic (distributed) gossip is available in academic literature, but the protocols used -- and in particular, their consequences -- are not easily understood intuitively.
To improve ease of understanding, this paper proposes a tool for visualising gossip graphs and working with arbitrary gossip protocols.
The goal of the tool is, then, to allow students and researchers to more quickly gain an understanding of the most important aspects of gossip theory.
\textcolor{red}{\textbf{Add:} Something about the results of the survey, something like ``To evaluate this goal, a survey was held among field experts. The survey results indicate users experience the tool as easy to use and useful in a study setting. Furthermore, feedback from the survey was used to improve the user experience and gain an understanding of which features are the most important to users.''}
\vspace{1ex}\\
\noindent\textbf{Keywords:} Dynamic Gossip, Elm
\end{abstract}