% !TEX root = report.tex

\section{Notation}\label{ap:a}

Let \(A\) be a set of agents \(\{a, b, \dots\}\).
Two binary relations on \(A\) are defined: \(N, S \subseteq A^2\).
The first denotes the \textit{number} relation, the second the \textit{secret} relation.
A gossip graph \(G\) is then defined as the a triple \((A, N, S)\).

\subsection{Gossip graphs}

\begin{description}
    \item[Binary relation] \(Pxy\)

        Agent \(x\) has relation \(P\) to agent \(y\).

        \(Pxy \equiv (x, y) \in P\)

    \item[Identity relation] \(I_A\)

        The relations of all agents in \(A\) with themselves.

        \(I_A = \{(x,x) \mid x \in A\}\)

    \item[Converse relation] \(P^{-1}\)

        The opposites of all relations in \(P\).

        \(P^{-1} = \{(x,y) \mid Pyx\}\)

    \item[Composition relation] \(P \circ Q\)

        The composition of the relations \(P\) and \(Q\) is a new relation such that the tuple \((x,z)\) is in said new relation iff there exists another agent y such that \((x,y) \in P\) and \((y,z) \in Q\)

        \(P \circ Q = \{(x,z) \mid \exists y ((x, y) \in P \land (y, z) \in Q) \}\)

    \item \(P_x\)
    
        The agents that agent \(x\) has a relation with.

        \(P_x = \{ y \in A \mid Pxy \}\)

    \item \(P^i\)

        The \(i\)th composition of relation \(P\) with itself

        \begin{math}
            P^i = 
            \begin{cases}
                P               & \text{for} \; i = 1\\
                P^{i-1} \circ P & \text{for} \; i > 1
            \end{cases}
        \end{math}
    
    \item \(P^*\)
    
        All binary relations that are possible through (repeated) relation composition of \(P\) with itself.

        \(P_x = \{ y \in A \mid Pxy \}\)

    
\end{description}