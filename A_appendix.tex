% !TEX root = report.tex

\section{Survey questions}\label{ap:survey}

\begin{enumerate}
    \item I am familiar with dynamic and/or distributed gossip.

    \textit{Sem. Diff. Scale: 1 = Not familiar at all , 5 = Very familiar}

    \item What do you think of the tool's visual design?

    \textit{Sem. Diff. Scale: 1 = Very bad, 5 = Very good}

    \item How much time does it take you to understand what each part of the interface is for?

    \textit{Sem. Diff. Scale: 1 = Very little time, 5 = Very much time}

    \item Do you think the error messages shown when providing incorrect input are helpful?

    \textit{Sem. Diff. Scale: 1 = Not helpful at all, 5 = Very helpful}

    \item Is there any part of the interface you think could use improvement? If so, please briefly explain which part and why.

    \textit{Text input field}

    \item Do you think this tool could be useful in a research setting?

    \textit{Yes/No/Maybe}

    \item Do you think this tool could be useful in a study setting?

    \textit{Yes/No/Maybe}

    \item Which features do you think are the most important in this tool?

    {\itshape
        List of options:
        \begin{itemize}
            \item Visualising gossip graphs;
            \item Checking permissibility of call sequences on gossip graphs under certain protocols;
            \item Executing call sequences on gossip graphs;
            \item Navigating between different states of a gossip graphs after (a) call(s) have been made;
            \item Gossip graph statistics: whether a graph is a sun graph, and whether a relation is strongly/weakly connected;
            \item *Creating custom protocols by creating a boolean combination of constituents (as defined in the paper "Dynamic Gossip" by van Ditmarsch et al. (2019));
            \item *Generating execution trees for gossip protocols;
            \item *Exporting gossip graphs to LaTeX or GraphViz formats;
            \item *Exporting execution tree diagrams to LaTeX or GraphViz formats;
            \item Other (Text input field).
        \end{itemize}

        The asterisk (*) indicates functionality that was not finished at the time the survey was held.
    }

    \item Are there any features that you would like to see added or changed?

    \textit{Text input field}
\end{enumerate}