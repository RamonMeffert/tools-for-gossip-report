% !TEX root = 0_report.tex

\section{Introduction}
\label{sec:introduction}

\textit{Gossip protocols} are protocols that describe the way rumours---or, more generally, secrets---are shared in multi-agent environments.
The goal of the protocols is to communicate all secrets to all agents.
A lot of research has been done in this field, 
starting with research on the spread of infectious diseases \parencite{kermack_contribution_1927}.

The origin of the gossip problem has proven to be hard to pinpoint exactly.
Most papers cite an unpublished work by R. Chesters and S. Silverman at the University of Witwatersrand.
This is most likely based on a review \parencite{graham_review_1972} of a paper by \textcite{tijdeman_telephone_1971}.
Tijdeman claims the problem was first formulated by A. Boyd, but Graham claims Chesters and Silverman proposed the original problem.
Another possibility is given by \textcite{lebensold_efficient_1973}, who claims the problem was proposed by Erd\H{o}s.
Although the paper by \citeauthor{tijdeman_telephone_1971} seems to be considered one of the first solutions to the gossip problem by many,
an unpublished paper by R.T.~Bumby and J.~Spencer might be earlier. 
\textcite{harary_communication_1974} cite this paper as being referenced in a work from 1966.

In short, agents are represented as nodes in a graph, with the edges representing a relation.
This relation can be either the number relation \(N\), representing that one agent has the ``phone number'' of another agent,
and the secret relation \(S\), representing that one agent knows the secret of another agent.
Many authors \parencite[e.g.][]{tijdeman_telephone_1971, hajnal_cure_1972, baker_gossips_1972, lebensold_efficient_1973} have proven that this can be done in no fewer than \(2n-4\) calls,
where \(n\) is the number of agents,
given that all agents know the numbers of all other agents---that is, the number relation \(N\) is \textit{complete}.

The problem as formulated above requires the oversight of a central authority in order to know whether all agents know all secrets.
However, there are many applications where this is not feasible or desirable as this becomes very computationally expensive as the number of agents increases.
Another problem is that it often cannot be guaranteed that all agents can contact all other agents.
This has led to the sub-fields of \textit{distributed gossip}, addressing the first issue, and \textit{dynamic gossip}, addressing the second.
The combination of these fields, where there is no overseer and not all agents can contact all other agents, is called \textit{distributed dynamic gossip}.

This paper will describe a tool to make it easier to explore distributed dynamic gossip.
It is able to visualise the connections between agents,
allows exploring the execution tree of different protocols and validate call sequences given a graph and/or a protocol.

\subsection{Earlier work}

\subsubsection{Applications of gossip}

Gossip, since its first formal description, found many different applications.
One of the earlier applications we could find was used for database consistency in the Xerox Corporate Internet \parencite{demers_epidemic_1988}.
More recent applications in system consistency can be found in the applications in Amazon's DynamoDB software \parencite{decandia_dynamo_2007} and the SWIM protocol \parencite{das_swim_2002}.
However, more applications than database consistency exist:
Notable examples include its use in genome research \parencite{liben-nowell_gossip_2002},
alternatives to the proof-of-work in blockchain \parencite{renesse_blockchain_2016,baird_swirlds_2016} and ad-hoc network routing \parencite{haas_gossip-based_2006,korecki_balancing_2020}.

\subsubsection{Visualisation tools}

A researcher or student interested in dynamic gossip will probably start looking for some way to gain an intuitive understanding of the field.
At the time of writing, a couple of applications that demonstrate properties of dynamic gossip are available.
However, most of these require either an established understanding of the field and/or proficiency using command-line tools.
In this section, these tools will be discussed in order to highlight some of the aspects the tool proposed in this paper attempts to improve.

The first two tools are tools by Malvin Gattinger (the supervisor of this Bachelor's project):
one works in the browser \parencite{gattinger_webgossip_2016}, the other is a command-line tool \parencite{gattinger_m4lvingomoche_2020}. 
The first tool, \texttt{WebGossip}, is able to visualise gossip graphs based on user input.
While the input format for that tool is different from the one used in this tool
(for an explanation of the format for our tool, see Section~\ref{sec:gossip-graph-visualisation}),
it is similar in that it also takes text input and displays the gossip graph graphically.
The tool also generates code for use in \LaTeX\ papers.

The second tool has more features but, being a command-line tool, is not as easy to use.
The main differences between these two tools and our tool can be seen in Table~\ref{tab:tool-comparison}


\begin{table*}
    \begin{threeparttable}
        \caption{Comparison between our tool and other tools. The tools by \textcite{maffre_faustinemaffregossipproblem-pddl-generator_2020} and \textcite{moelker_rrmoelkergossip-visualization_2016} have not been included in this comparison since they are too different for useful comparison.}
        \label{tab:tool-comparison}
        \begin{tabular}{@{}p{0.3\linewidth}*{3}{M{0.2\linewidth}}@{}}
            \toprule
            Feature 
                & \texttt{GoMoChe} 
                & \texttt{WebGossip} 
                & This paper\\
            \midrule
            Interface
                & Command line
                & Web
                & Web
                \\
            Gossip graph visualisation
                & \tpos{Yes}
                & \tpos{Yes}
                & \tpos{Yes}
                \\
            \LaTeX\ output
                & \tneg{No}
                & \tpos{Yes}
                & \tmay{Planned}
                \\
            Call execution
                & \tneg{No}
                & \tneg{No}
                & \tpos{Yes}
                \\
            Call sequence execution
                & \tneg{No}
                & \tneg{No}
                & \tpos{Yes}
                \\
            Show allowed calls
                & \tpos{No}
                & \tneg{No}
                & \tpos{Yes}
                \\
            Show allowed sequences
                & \tpos{Yes}
                & \tneg{No}
                & \tneg{No}
                \\
            Show execution tree
                & \tneg{No}
                & \tneg{No}
                & \tpos{Yes}
                \\
            Protocols
                & LNS, CO\tnote{1}, COwLoG, PIG, ANY, NO
                & N/A
                & LNS, CO, wCO, SPI, ANY, TOK, \emph{custom}
                \\
            \bottomrule
        \end{tabular}
        \begin{tablenotes}
            \item[1] Call-me-once; sometimes also called CMO.
        \end{tablenotes}
    \end{threeparttable}
\end{table*}

Two other tools worth mentioning are by \textcite{maffre_faustinemaffregossipproblem-pddl-generator_2020} and \textcite{moelker_rrmoelkergossip-visualization_2016}.
The first allows users to generate domain and problem files in the Planning Domain Definition Language (PDDL) for a given gossip protocol, making it possible to analyse protocol execution using PDDL software.
The second is a visualisation tool, but due to the lack of a description on the project page, it is not entirely clear what the tool does. It seems to implement some specific gossip protocol and simulate interactions between agents.

In conclusion, very few tools are available for exploring dynamic gossip.
The tools that \emph{are} available are either lacking in features or provide an uninviting user interface.
Our tool 

\subsection{Notation}
\label{sec:notation}

This paper takes its notation from \Textcite{van_ditmarsch_dynamic_2018} and \Textcite{van_ditmarsch_strengthening_2019}.
The notation used is explained below.

\begin{definition}[Gossip graph]
    Let \(A\) be a set of agents \(\{a, b, \dots\}\).
    Two directed binary relations on \(A\) are defined: \(N, S \subseteq A^2\).
    The first denotes the \textit{number} relation, the second the \textit{secret} relation.
    A gossip graph \(G\) is then defined as a triple \((A, N, S)\).
\end{definition}

\begin{definition}[Relations on gossip graphs]
    
\begin{subdefinition}[Binary relation]
    When agent \(x\) has relation \(P\) to agent \(y\), this is denoted as \(Pxy\). 
    A binary relation is notated as a tuple \((x,y)\) and represents a directed relation from \(x\) to \(y\). 
    
    Formally: \(Pxy := \{ (x, y) \mid (x,y) \in P \}\)
    \label{def:rel-bin}
\end{subdefinition}

\begin{subdefinition}[Identity relation]
    The set of relations where all agents have a relation to themselves is denoted \(I_A\).
    
    Formally: \(I_A := \{(x,x) \mid x \in A\}\)
    \label{def:rel-id}
\end{subdefinition}

\begin{subdefinition}[Converse relation]
    The set of reverse relations in \(P\) is denoted \(P^{-1}\).

    Formally: \(P^{-1} := \{(x,y) \mid Pyx\}\)
    \label{def:rel-conv}
\end{subdefinition}

\begin{subdefinition}[Composition relation]
    The composition of the relations \(P\) and \(Q\) is a new relation such that the tuple \((x,z)\) is in said new relation if and only if there exists another agent y such that \((x,y) \in P\) and \((y,z) \in Q\).

    Formally: \(P \circ Q := \{(x,z) \mid \exists y ((x, y) \in P \land (y, z) \in Q) \}\)
    \label{def:rel-comp}
\end{subdefinition}

\begin{subdefinition}
    The set of agents \(x\) has relation \(P\) with is denoted \(P_x\).

    Formally: \(P_x := \{ y \in A \mid Pxy \}\)
    \label{def:rel-acq} % acq for acquaintances ('acquaintance relation' might be a nice name for this relation)
\end{subdefinition}

\begin{subdefinition}
    The \(i\)th composition of relation \(P\) with itself is denoted \(P^i\).

    Formally: 
    \[
        P^i := 
        \begin{cases}
            P               & \text{for} \; i = 1\\
            P^{i-1} \circ P & \text{for} \; i > 1
        \end{cases}
    \]
    \label{def:rel-icomp}
\end{subdefinition}

\begin{subdefinition}
    The transitive closure of \(P\) is denoted \(P^*\).

    Formally:
    \[
        P^* := \bigcup_{i \geq 1} P^i
    \]
    \label{def:rel-star} % not sure what to call this relation
\end{subdefinition}
\end{definition}

\begin{definition}[Properties of gossip graphs]
    \begin{subdefinition}[Weakly connected]
        A graph \(G\) if for all \(x, y \ in A\) there exists a number relation from \(x\) to \(y\) or from \(y\) to x, that is, the symmetric closure of \(N\) contains \((x, y)\).
        
        Formally: 
            \[
                A^2 = \{ (x, y) \in (N \cup N^{-1}) \mid x, y \in A \}
            \]
        \label{def:weakly-connected}
    \end{subdefinition}
    
    \begin{subdefinition}[Strongly connected]
        A graph \(G\) is strongly connected if for all \(x, y \ in A\) there exists a number \(N\) relation from \(x\) to \(y\).
        
        Formally:
            \[
                A^2 = \{ (x, y) \in N \mid x, y \in A \}
            \]
        \label{def:strongly-connected}
    \end{subdefinition}
    
    \begin{subdefinition}[Sun graph]
        A graph \(G\) is a sun graph iff its non-terminal nodes are strongly connected.
        \label{def:sun-graph}
    \end{subdefinition}
\end{definition}

\begin{definition}[Calls]
    \label{def:calls}
    A call from an agent \(x\) to an agent y is denoted \(xy\).

    \begin{subdefinition}[Call sequence]
        \label{def:call-sequence}
        Call sequences are denoted using lowercase greek letters, starting with \(\sigma\). The empty sequence is denoted \(\epsilon\) and \(\sigma \conc \tau\) denotes the concatenation of two sequences.
    \end{subdefinition}
    
    \begin{subdefinition}[Subsequence]
        The set of all calls in a sequence \(\sigma\) containing \(x\) is denoted \(\sigma_x\).

        Formally:
        \[
            \sigma_x := 
            \begin{cases} 
                \epsilon            & \text{if}\; \sigma = \epsilon\\
                \tau_x\mathbin{;}uv & \text{if}\; \sigma = \tau \conc uv \land (x = u \lor x = v) \\ 
                \tau_x              & \text{if}\; \sigma = \tau \conc uv \land \lnot(x = u \lor x = v) 
            \end{cases}
        \]
    \end{subdefinition}

    \begin{subdefinition}[Effect of a call on a relation]
        \label{def:call-effect-relation}
        The state of relation \(P\) after call \(xy\) is denoted \(P^{xy}\).
        % TODO: Give an example
        Formally: \( P^{xy} := P \cup (\{(x,y),(y,x)\} \circ P) \)
    \end{subdefinition}

    \begin{subdefinition}[Effect of a call on a gossip graph]
        \label{def:call-effect-graph}
        The state of gossip graph \(G\) after call \(xy\) is denoted \(G^{xy}\).

        Formally: \( G^{xy} := (A,N^{xy},S^{xy}) \)
    \end{subdefinition}

    \begin{subdefinition}[Effect of a call sequence on a gossip graph]
        The state of gossip graph \(G\) after call sequence \(\sigma\) is denoted \(G^\sigma\).

        Formally: 
        \[ 
            G^\sigma := 
            \begin{cases}
                G               & \text{if}\; \sigma = \epsilon\\
                (G^{xy})^\tau   & \text{if}\; \sigma = xy \conc \tau
            \end{cases}
        \]
    \end{subdefinition}

    \begin{subdefinition}[Call sequence possibility]
        \label{def:call-sequence-possibility}
        A call sequence \(\sigma\) is \emph{possible} on a gossip graph \(G\) iff for all calls \(c_i \in \sigma\) such that \(\sigma = c_i\mathbin{;}\tau\), the call is possible, i.e. \(Nc_i\)
    \end{subdefinition}

    \begin{subdefinition}[Call permissibility]
        \label{def:call-permissibility}
        A call \(xy\) is \emph{permitted} under protocol condition \(\pi(x,y)\) on a gossip graph \(G\) iff \(x \neq y\), \(N xy\) and \(G \models \pi(x,y)\).
    \end{subdefinition}

    \begin{subdefinition}[Call sequence permissibility]
        \label{def:call-sequence-permissibility}
        A call sequence \(\sigma \conc xy\) is permitted on a gossip graph \(G\) iff \(xy\) is permitted on \(G^\sigma\) and \(\sigma\) is permitted on \(G\).
    \end{subdefinition}
\end{definition}

\begin{definition}[Gossip Protocols]
    \label{def:gossip-protocols}
    \begin{subdefinition}[Protocol condition]
        \label{def:protocol-condition}
        \(\pi(x,y)\) represents a protocol condition, that is, a condition that needs to be satisfied before a call can be made.
        This report takes this condition, like \Textcite{van_ditmarsch_dynamic_2018}, as a boolean combination of the following constituents: \(S^\sigma xy\), \(xy \in \sigma_x\), \(yx \in \sigma_x\), \(\sigma_x = \epsilon\), \(\sigma_x = t \conc xz\) and \(\sigma_x = t \conc zx\).
    \end{subdefinition}
    
    \begin{subdefinition}[Gossip Protocol]
        \label{def:gossip-protocol}
        A gossip protocol \(\mathsf{P}\) is a protocol condition \(\pi(x,y)\) together with a non-deterministic algorithm of the following form:
        
        \begin{displayquote}
            Until all agents are experts, select \( x, y \in A \), such that \( x \neq y \), \(Nxy\), and \( \pi(x, y) \), and execute call \(xy\).
            
        \parencite[Def. 5]{van_ditmarsch_dynamic_2018}.
        \end{displayquote}
        
    \end{subdefinition}
\end{definition}

% TODO: finish writing the notation. I have a notes file containing all notation, just have to convert it to latex

% - gossip protocols
% - reference some papers indicating origin of problem
% - explain dynamic gossip
% - examples of applications
% - rationale for building a tool
%   - visualise protocol execution
%   - exploration of new protocols
