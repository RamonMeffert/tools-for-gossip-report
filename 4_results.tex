% !TEX root = 0_report.tex

\section{Results}

The survey was open for responses during the period between December 19, 2020 and January 13, 2021 and received 12 responses (N = 12).
A Wilcoxon Rank Sum test showed a significant non-neutral response on questions about familiarity with gossip (median = 4, V = 45, p = 0.007), interface design (median = 4.5, V = 78, p = 0.002) and usefullness of error messages (median = 4, V = 74, p = 0.005).
Users did not require significantly low or high amounts of time before they understood what each part of the interface was for (median = 3, V = 6, p = 0.19).
The majority of users thought the tool could be useful in a study setting, but the response for a research setting was more divided (Table~\ref{tab:setting-usefulness}).

\begin{table}[htb!]
    \centering
    \caption{Usefulness of the tool in a research or study setting.}
    \label{tab:setting-usefulness}
    \begin{tabular}{lccc}
        \toprule
                    & \multicolumn{3}{c}{Response}          \\\cmidrule{2-4}
        Setting     & No        & Maybe     & Yes           \\
        \midrule
        Research    & 1 (8\%)   & 5 (33\%)  & 6 (50\%)      \\
        Study       & 0 (0\%)   & 0 (0\%)   & 12 (100\%)    \\
        \bottomrule
    \end{tabular}
\end{table}

Question 8 (Importance of features) resulted in the following order of importance (the number between parentheses represents the number of responses):

\begin{enumerate}
    \item[1] (10) Checking permissibility of call sequences on gossip graphs under certain protocols	
    \item[ ] (10) Visualising gossip graphs	
    \item[2] (9)  Executing call sequences on gossip graphs	
    \item[3] (8)  *Exporting gossip graphs to LaTeX or GraphViz formats	
    \item[4] (6)  *Creating custom protocols by creating a boolean combination of constituents (as defined in the paper ``Dynamic Gossip'' by van Ditmarsch et al. (2019))	
    \item[ ] (6)  Gossip graph statistics: whether a graph is a sun graph, and whether a relation is strongly/weakly connected	
    \item[5] (5)  *Exporting execution tree diagrams to LaTeX or GraphViz formats	
    \item[6] (4)  *Generating execution trees for gossip protocols	
    \item[ ] (4)  Navigating between different states of a gossip graphs after (a) call(s) have been made
\end{enumerate}