% !TEX root = report.tex

\section{Method}
\label{sec:method}

% TODO: Rewrite this paragraph so it has better flow
To provide as much useful information as possible, the tool needs to provide information on a number of aspects of gossip.
The first aspect that will be discussed is gossip graph visualisation, followed by protocol execution, call sequence validation and call (sequence) execution.
First, a reasoning will be given behind the programming language of choice.
Then, these aspects are discussed in terms of their meaning in both the context of gossip theory and of their implementation.
Lastly, some time will be spent to explain how these aspects are integrated into the application.

\subsection{Implementation}

This project uses Elm, a statically typed functional programming language for web development with its roots in Functional Reactive Programming \parencite{czaplicki_asynchronous_2013}.
This has several advantages:

\begin{enumerate}
    \item Implementing mathematical functions is more natural in functional languages because functions in Elm are pure;
    \item Elm is compiled to standard Javascript. Since all modern browsers support Javascript, this ensures the application is cross-platform;
    \item Elm does static type checking while compiling, ensuring type safety and thus reducing runtime errors;
\end{enumerate}

Another advantage of using a functional language is that much of the notation introduced in section~\ref{sec:notation} can be translated fairly directly.
For example, to evaluate protocol conditions, the last call in a call sequence needs to be checked.
In mathematical notation this is represented as \(\sigma_x = \tau \conc xy\) 
(``the last call in the sequence of calls containing \(x\) was a call made by \(x\) to another agent \(y\)'').
This can be represented in Elm quite naturally, as can be seen in Listing~\ref{lst:elm-ex-1}.

\begin{lstlisting}[caption={\(\sigma_x = \tau \conc xy\) in Elm.}, label=lst:elm-ex-1]
lastFrom agent sigma_x = 
    case reverse sigma_x of
        [] ->
            False

        (x, y) :: tau ->
            x == agent
\end{lstlisting}

This states that if \(\sigma_x = \epsilon\), the condition is false. 
Otherwise, it checks to see if the last call was made by \(x\), and returns true if that is the case.

\subsubsection{Gossip graph visualisation}\label{sec:gossip-graph-visualisation}

One of the easiest ways to visualise connections in a network is by using a graph.
Therefore, the application is able to visualise gossip graphs based on user input.
The input is provided by users in text, and uses a format based on the one used in the appendix of \textcite{van_ditmarsch_strengthening_2019}.
However, since no complete grammar of that format is given,
the format has been adapted to be a bit more permitting than could be expected based purely on its description.

The basic idea behind the format is that every agent is represented by some letters, which in turn represent agent names.
Uppercase letters represent the Secret relation \(S\) and lowercase letters represent the Number relation \(N\).
Because every agent knows their own secret (\(I_A \subseteq S\)), it is possible to find names for all agents if and only if the input represents a valid gossip graph.
That is, every agent is represented by some letter segment such that the number of segments is equal to the number of unique agent names, and every letter segment contains a secret relation that allows it to be uniquely identified.
This procedure is described in more detail in Algorithm~\ref{alg:find-agents}.

\begin{algorithm*}[p]
    \SetKwInOut{Input}{input}
    \SetKwInOut{Output}{output}
    \SetKwFunction{Up}{GetUppercaseLetters}
    \SetKwFunction{Error}{ShowError}
    \SetKwData{Names}{names}
    \SetKwData{Segments}{segments}
    \SetKwData{Segment}{segment}
    \SetKwData{PosNames}{possibleNames}
    \SetKwData{Name}{name}
    \SetKwData{Unique}{UniqueNameFound}
    \DontPrintSemicolon

    \Input{A list of letter sequences called \Segments}
    \Output{A list of agent names}
    \Names $\leftarrow$ \{\}\;
    \ForEach{\Segment $\in$ \Segments}{
        \PosNames $\leftarrow$ \Up{\Segment}\;
        \Unique $\leftarrow$ false\;
        \ForEach{\Name $\in$ \PosNames}{
            \If{\Name $\notin$ \Names}{
                \Names $\leftarrow$ \{\Name\} $\cup$ \Names\;
                \Unique $\leftarrow$ true\;
            }
        }
        \BlankLine
        \If{$\neg $\Unique}{
            \Error{}\;
        }
    }
    \BlankLine
    \Return{\Names}\;
    \caption{Finding agent names}\label{alg:find-agents}
\end{algorithm*}

Once the agent names have been found, it is possible to parse the relations from the input string.
Algorithm~\ref{alg:find-agents} assumes the first agent name it finds represents the first agent. 
Thus, the first segment represents the relations of that agent.

% more info

This interpretation of the format is likely a bit more permissive than the original format.
For example, in the original format, the names of agents always follow alphabetic order -- agent 1 is A, agent 2 is B, et cetera.
Furthermore, the text segments in the original format are always ordered alphabetically.
These requirements are not present in the implementation, so an input like \texttt{Xyz Yzx Zxy} is allowed.
However, the tool does generate a so-called ``canonical input'' which does follow these rules.

%TODO: Add paragraph about how graph is updated whenever the input changes and that helpful error messages are presented when the input is not valid.

\subsubsection{Execution calls}

\paragraph{Single calls}
\label{sec:call-execution}
When a gossip graph is given by the user, the tool allows them to execute calls and call sequences. 
Because the possible calls depend on the active protocol,
a list of possible calls is given for the currently active protocol.
The possible calls are found by checking for call permissibility as defined in Definition~\ref{def:call-permissibility}.
When the user clicks one of these calls, the call is executed.
The call then happens according to Definition~\ref{def:call-effect-graph}.

\paragraph{Call sequences}
It is also possible to execute entire call sequences.
The user is able to enter a call sequence using a text input according to the format given in Definition~\ref{def:call-sequence}.
This sequence is then checked for possibility (Definition~\ref{def:call-sequence-possibility}) and permissibility (Definition~\ref{def:call-sequence-permissibility}).
It should be noted that these checks are done in the order mentioned here.
This means that the user receives feedback on their input when they enter a call sequence that is not possible on the current gossip graph.
This has the added advantage that the system need not check for permissibility if the user enters a call sequence that is impossible.
Additionally, because possibility is a requirement for permissibility and its check is executed before the permissibility check, this check does not need to occur in the check for permissibility;
if the system reaches the permissibility check, the call sequence is guaranteed to be possible.

\paragraph{Custom protocols}
\label{sec:custom-protocols}
To allow more flexibility in using protocols, 
the tool described in this paper allows the user to create their own gossip protocols.
This is done through a drag-and-drop interface,
in which users can arrange boolean constituents or groups of boolean constituents.
These constituents are then interspersed with either logical conjunction (\(\land\)) or logical disjunction (\(\lor\)) (see also Figure~\ref{fig:protocol-creator}).

\begin{figure}[htb!]
    \centering
    \includegraphics[width=\linewidth]{example-image-1x1}
    \caption{The interface for creating custom gossip protocols (placeholder)}
    \label{fig:protocol-creator}
\end{figure}

The creation of custom protocols is based on the language \(\mathcal{L}_B(V)\) of boolean formulas, 
given in the Backus-Naur form:
\(\varphi ::= \top \mid p \mid \neg \varphi \mid \varphi \land \varphi\), where \(p \in V\). This is extended with the logical disjunction by virtue of the formula \(\varphi \lor \psi := \neg(\neg\varphi \land \neg\psi)\).
For this tool, the vocabulary \(V\) is the set of constituents of protocol conditions as defined in Definition~\ref{def:protocol-condition}.

\subsubsection{Execution tree and call history}

Calls are recorded and it is possible to time travel between states.
\textcolor{red}{Something about multiple branches}

\subsection{Evaluation}

Survey blah blah blah